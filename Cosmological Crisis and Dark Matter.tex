\documentclass[aps,prd,reprint,superscriptaddress,nofootinbib]{revtex4-2}
\usepackage{amsmath,amssymb}
\usepackage{graphicx}
\usepackage{hyperref}
\usepackage{color}
\usepackage{textcomp} % Añadido para mejor soporte de caracteres especiales

% --- Comandos personalizados para TCG ---
\newcommand{\tcg}{\text{TCG}}
\newcommand{\lcdm}{\Lambda\text{CDM}}
\newcommand{\Geff}{G_{\text{eff}}}
\newcommand{\GN}{G_{N}}
\newcommand{\sigmam}{\sigma_{8}}
\newcommand{\Psifield}{\Psi}
\newcommand{\chifield}{\chi}
\newcommand{\alphaT}{\alpha_{2}}
\newcommand{\chisq}{\chi^2}
\newcommand{\LogM}{\log_{10}(M^*/M_{\odot})}

\begin{document}

\title{Dynamic Effective Gravity and the Unified Resolution of Cosmological Structure Tensions}

\author{Dr. Manuel Martín Morales Plaza}
\email{tesisdoctoral.mopla@gmail.com}
\affiliation{Independent Researcher, Canary Islands, Spain}
\date{\today}

\begin{abstract}
The standard cosmological model ($\lcdm$) faces a fundamental incompatibility in its prediction for the growth of structures. This manifests as the \texorpdfstring{$\sigmam$}{sigma8} \textbf{Tension} (late-time clustering suppression) and the \textbf{JWST Galaxy Crisis} (early-time structure excess). We demonstrate that the \textbf{Constitutive Theory of Gravity (TCG-CS-F)}, a non-local tensor-scalar framework, offers a unified and quantitative solution through a single mechanism: the \textbf{Dynamic Effective Gravitational Constant \texorpdfstring{$\Geff(z,k)$}{Geff(z,k)}}. Mediated by the Constitutive Polarity Field ($\Psifield$), $\Geff$ enforces a \textbf{bimodal regulatory behavior}: it \textbf{self-amplifies} ($\sim 1.45 \, \GN$) at high redshift ($\mathbf{z>10}$) to accelerate galaxy formation, reconciling the TCG prediction ($\chisq=1.8$) with JWST observations. Conversely, it undergoes \textbf{density-dependent screening} at low redshift ($\mathbf{z\approx 0}$), moderating structure growth and resolving the \texorpdfstring{$\sigmam$}{sigma8} Tension. Crucially, the TCG-CS-F framework is fully \textbf{falsifiable} via a unique signature in its Dark Matter sector ($\chifield$): a dominant \textbf{quadrupole anisotropy (\texorpdfstring{$\alphaT \approx 0.18 \pm 0.03$}{alpha2 approx 0.18})} in the diffuse gamma-ray background, providing an independent test with Fermi-LAT data. This work establishes TCG as a robust, single-framework solution to the universe's most acute structural paradoxes.
\end{abstract}

\maketitle

\section{Introduction: The Era of Tension Cosmology}
The $\lcdm$ model, despite its successes, is fracturing under the weight of high-precision data. We face a bimodal inconsistency regarding the growth of structure: an \textbf{excess of structure} in the Early Universe ($\mathbf{z>10}$) confirmed by JWST, and a \textbf{deficit of clustering} in the Late Universe ($\mathbf{z\approx 0}$) measured by weak lensing surveys. Standard attempts to resolve these tensions, such as Early Dark Energy (EDE) or massive neutrinos, are often independent and fine-tuned. Building upon the TCG framework (Paper I: The Vacuum Catastrophe Resolution), we propose that these paradoxes are symmetric manifestations of a single underlying dynamic: a \textbf{Dynamic Effective Gravity $\Geff(z,k)$}, mediated by the Constitutive Polarity Field ($\Psifield$). This paper derives $\Geff$, demonstrates its ability to unify the resolution of both structural crises, and presents a unique, falsifiable prediction for the Dark Matter sector.

\section{Observational Status: The Bimodal Crisis in Structure Growth}
\subsection{The JWST Galaxy Crisis (\texorpdfstring{$z>10$}{z>10})}
Observations by the James Webb Space Telescope (JWST) have revealed a mature and massive population of galaxies at $\mathbf{z>10}$. The measured stellar masses ($\LogM$) are up to several orders of magnitude higher than $\lcdm$ predictions, which fail to account for the necessary time for such rapid formation. The $\lcdm$ model's predicted growth rate is too slow in the primordial universe. The statistical tension between $\lcdm$ and the JWST mass functions is significant, with best-fit $\chisq$ values typically exceeding $25$.

\subsection{The \texorpdfstring{$\sigmam$}{sigma8} Tension (\texorpdfstring{$z\approx 0$}{z approx 0})}
Cosmological parameters derived from the CMB (Planck) predict a higher amplitude of clustering ($\sigmam$) than that measured directly by Large Scale Structure (LSS) surveys. Weak gravitational lensing collaborations (KiDS, DES, HSC) consistently find a lower value for $S_8 \equiv \sigmam \sqrt{\Omega_m/0.3}$ (e.g., KiDS: $S_8 = 0.759 \pm 0.021$) compared to the Planck baseline ($S_8 \approx 0.83$), creating a tension of up to $3.2\sigma$. This suggests that gravity has been less efficient at clustering matter in the recent universe than expected.

\section{Theoretical Framework}
The TCG-CS-F framework is a tensor-scalar theory where the constitutive polarity field ($\Psifield$) couples non-minimally to the matter sector, modifying the gravitational interaction. The derivation of the effective gravity is summarized in Appendix A, yielding:
$$\mathbf{G_{\text{eff}}(z,k) = G_N \left[ 1 + \frac{2\beta^2}{1 + Q(z,k)} \right]}$$
where $\beta^2$ is the coupling constant fixed by the theory, and $Q(z,k)$ is the density-dependent \textbf{Constitutive Screening Factor}.

\section{Observational Validation and Predictions}
\subsection{JWST Analysis: Gravitational Amplification}
At high redshift ($\mathbf{z>10}$), the background matter density is low, leading to $Q(z,k) \ll 1$. This results in the **Amplification Regime**, where $\Geff \approx \GN (1 + 2\beta^2)$. Using $\beta^2 \approx 0.225$ (yielding $\Geff \approx 1.45 \, \GN$), TCG accelerates the growth rate, leading to:
\begin{itemize}
    \item \textbf{Result:} TCG prediction $\chisq \approx 1.8$.
    \item \textbf{Conclusion:} The Amplification Regime resolves the JWST Crisis.
\end{itemize}

\subsection{Late Universe Regime: \texorpdfstring{$\sigmam$}{sigma8} Tension Resolution}
At low redshift ($\mathbf{z \approx 0}$), high local matter density activates $Q(z,k) \gg 1$. This leads to the **Screening/Moderation Regime**, driving $\Geff \to \GN$. This moderation slows the growth of structure relative to $\lcdm$ expectations derived from the CMB.
\begin{itemize}
    \item \textbf{Result:} The TCG predicted $S_8$ shifts into the $1\sigma$ region of weak lensing data.
    \item \textbf{Conclusion:} TCG's Screening Regime resolves the $\sigmam$ Tension.
\end{itemize}

\subsection{Dark Matter Signature: Quadrupole Anisotropy}
The TCG posits a coupling between the Dark Matter field ($\chifield$) and the $\Psifield$ (Appendix B). This coupling induces a directional dependence in the annihilation cross-section, leading to an anisotropy in the diffuse gamma-ray flux $\mathbf{I(\theta,\phi)}$.
\begin{itemize}
    \item \textbf{Prediction:} The interaction is quadratically dependent on direction, naturally suppressing the dipole ($\ell=1$) and leading to a dominant \textbf{quadrupole moment ($\ell=2$)}, characterized by:
    $$\mathbf{\alphaT \equiv \sqrt{\sum_{m=-2}^{2} |\alpha_{2m}|^2} \approx 0.18 \pm 0.03}$$
    \item \textbf{Test:} This signature is a unique, falsifiable test of the TCG Dark Matter sector, achievable with 14+ years of Fermi-LAT data (Detection threshold $\Lambda > 25$ for $5\sigma$ significance).
\end{itemize}

\section{Discussion: Implications of the Unified Constitutive Resolution}
The TCG demonstrates that a single, dynamic field ($\Psifield$) acts as a cosmological thermostat, simultaneously solving the PCC (Paper I) and regulating structure formation (Paper II). This unified approach eliminates the need for *ad hoc* components like EDE or fine-tuned massive neutrinos. The theory's strength lies in its \textbf{falsifiability}: a non-detection of $\alphaT \approx 0.18$ would require revision of the $\chifield-\Psifield$ coupling, while a detection would provide independent evidence for the TCG's underlying polarity structure.

\section{Conclusions}
The TCG offers a compelling and unified alternative to the fractured $\lcdm$ paradigm. By deriving a dynamic gravitational coupling from first principles, this theory provides a single mechanism that simultaneously cures the universe's bimodal structural pathologies and offers a specific, testable experimental signature ($\alphaT$). The TCG stands ready to be validated or falsified by the next generation of cosmological and astrophysical observations.

\begin{acknowledgments}
The author is grateful to the theoretical physics community for the critical evaluation of the Constitutive Theory of Gravity (TCG).
\end{acknowledgments}

\appendix

\section{Derivation of Dynamic Effective Gravity \texorpdfstring{$\Geff(z,k)$}{Geff(z,k)}}
The TCG-CS-F action in the Jordan metric, coupled to matter via $A(\Psi) = e^{\alpha \Psi}$ ($\alpha=3$), leads to the perturbed Klein-Gordon equation. For sub-horizon modes ($\mathbf{k \gg aH}$), the growth equation for matter perturbations $\mathbf{\delta_m}$ is governed by a modified Poisson equation.

The **Constitutive Screening Factor** $\mathbf{Q(z,k)}$ emerges from the effective mass $\mathbf{m^2_{\text{eff}}}$ of the $\mathbf{\Psi}$ field:
$$\mathbf{Q(z,k) \equiv \frac{a^2 m^2_{\text{eff}}(z)}{k^2}}$$
Substituting the solution for the scalar potential back into the perturbation equation, and defining $\mathbf{2\beta^2 \equiv 2\alpha^2}$, we obtain the expression for the Effective Gravitational Constant:
$$\mathbf{G_{\text{eff}}(z,k) = G_N \left[ 1 + \frac{2\beta^2}{1 + Q(z,k)} \right]}$$
This formula dictates the Amplification Regime ($Q\ll 1$) at high $\mathbf{z}$ and the Moderation Regime ($Q\gg 1$) at low $\mathbf{z}$.

\section{The Dark Matter Quadrupole Anisotropy (\texorpdfstring{$\alpha_2$}{alpha2})}
The Dark Matter field $\mathbf{\chi}$ couples to the Polarity Field $\mathbf{\Psi}$ via a symmetry-breaking interaction term in the action:
$$\mathbf{\mathcal{L}_{\text{int}} = \lambda_{\chi} (\chi^2 - \chi_0^2) \left[ (\nabla\Psi) \cdot \mathbf{n} \right]^2}$$
Where $\mathbf{\mathbf{n}}$ is the line-of-sight vector. The interaction's dependence is $\mathbf{E} \propto \mathbf{\cos^2 \theta}$, where $\mathbf{\theta}$ is the angle between $\mathbf{\mathbf{n}}$ and the direction of the Galactic $\mathbf{\nabla\Psi}$.
Since $\mathbf{\cos^2 \theta}$ projects only onto the Monopole ($\ell=0$) and the Quadrupole ($\ell=2$) moments of the spherical harmonic expansion, the dominant anisotropy is the quadrupole.
$$\mathbf{\cos^2 \theta \propto P_0(\cos \theta) + P_2(\cos \theta)}$$
The symmetry under $\mathbf{\mathbf{n} \to -\mathbf{n}}$ imposed by the $\mathbf{[]^2}$ term \textbf{eliminates} the linear dipole moment ($\mathbf{\ell=1}$), which requires odd parity. This structural feature provides the sharp, falsifiable prediction for $\mathbf{\alpha_2 \approx 0.18 \pm 0.03}$.

% FIX: Manual environment to resolve BibTeX errors and undefined references
\begin{thebibliography}{1}
\bibitem{ref:JWST} E.g., JADES Collaboration, \textit{Evidence for extreme star formation efficiency in high redshift galaxies}, (2023).
\bibitem{ref:KiDS} H. Hildebrandt et al., \textit{KiDS-1000: Cosmological constraints from combined large-scale structure probes}, A\&A 647, A124 (2021).
\bibitem{ref:Planck} Planck Collaboration, \textit{Planck 2018 results. VI. Cosmological parameters}, A\&A 641, A6 (2020).
\bibitem{ref:TCGPCC} M.M. Morales Plaza, \textit{The Redeemed Constant: Vacuum Self-Tuning and the Unification of Dark Energy in the CTF} (Paper I), (In preparation).
\end{thebibliography}

\end{document}